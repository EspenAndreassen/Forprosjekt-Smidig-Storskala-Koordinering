\chapter{Introduction}

\minitoc

\newpage

\section{Motivation}

I am now entering my last year on a master degree in computer science where I specialise in software, or more specifically, software systems. I was introduced to agile development methodologies through different subjects at the ``Norwegian University of Science and Technology'', NTNU, and also got hands-on experience working with Scrum in a subject called ``TDT4290 - Customer Driven Project''. This subject in particular sparked my interest in agile development methodologies and the new ways of handling work and project organisation. After a summer internship with EY (former known as Ernst\&Young) I got more intrigued with how communication and coordination was handled in real life business and IT projects. Therefore, my previous experiences led to a motivation in exploring the combination of agile development and coordination.

\section{Problem Description and Background}

%TODO: Se om man skal ha med slutten (se på andre felter og sånt biten)

Since the introduction of agile development methodologies their usage have seen a steady growth. This has led to an increasing need for studies that reflect on the consequences and different aspects following the paradigm shift. One of these aspects is how coordination is handled \cite{Agerfalk2006, Leffingwell2007, Cockburn2002, Batra2010}. At the International Conference on Agile Software Development (XP2013) ``Inter-team coordination'' was voted the number one burning topic in large-scale agile software development, with ``Large project organization'' coming in second \cite{Dingsoyr2013b}.  In the latest years there has evidentially been an increase in companies and organisations performing development through agile development methodologies in large-scale projects \cite{Paasivaara2012, Com2013, Vlietland2015, Lindvall2004, Dingsoyr2013b, Lee2008, Paasivaara2009}, but the effects have not been well-documented \cite{Pikkarainen2008, Paasivaara2012, Freudenberg2010, Haaster2014, Dingsoyr2013a, Reifer2003}. In my study this topic will be highlighted with the focus on coordination in large-scale agile projects. Theories, literature and frameworks from the Software-field will be used and compared to other fields to see which changes and similarities the paradigm shift has brought forth (theories and literature from large-scale will be used where this is available).

Further, the Software-field especially has moved towards a higher degree of uncertainty and chaos, mainly because of the urge to be first-to-market and technology in constant change. This has produced an increasing need for flexibility in every stage of production and development \cite{Pikkarainen2008, Beck2004, Borjesson2004}. The combination of the increasing need for flexibility, as well as the acknowledgement of effective coordination as an important part of organisations and their projects has led to the research question:

\begin{fancyquotes}
How does coordination affect the level of efficiency achieved in large-scale agile projects?
\end{fancyquotes}

%TODO: Se mer på denne

The purpose of the study and the planned master thesis will therefore be a combination of ``To add to the body of knowledge'', ``To solve a problem'', ``To find the evidence to inform practice'', ''To develop a greater understanding of people and their world'' and ``To contribute to other people's well-being'' \cite{Oates2006}.

While research in small-scale agile software development is starting to get a good track record \cite{Paasivaara2012, Haaster2014}, there is a clear gap in the research surrounding coordination in large-scale agile software development \cite{Pikkarainen2008, Paasivaara2012, Dingsoyr2013b}, and large-scale agile software development in general \cite{Freudenberg2010, Haaster2014}. Therefore, this literature study, as well as a planned master thesis, will contribute in filling parts of the gap. This will involve ``An exploration of a topic, area or field'', as well as ``An in-depth study of a particular situation'' in the case study planned for the master thesis \cite{Oates2006}.

As stated above, small-scale agile software development research is starting to get a good track record with successful findings. Because of these findings large organisations have been interested in adopting the benefits agile software development has shown over traditional development methods \cite{Com2013, Vlietland2015, Agerfalk2006, Paasivaara2012}. The assumption that agile methodologies will deliver the same benefits when scaled to larger organisations and projects is therefore an interesting topic.

The combination of filling the gap and looking at the aforementioned assumption will be the pillars in the research outcomes.

\section{Scope and Limitations}

%Pga tids-constraints
%Coordination (and large-scale) som snakket om over, ikke alle andre områder som har med agile og large-scale å gjøre
%Ikke komme med ny teori, men heller et studie for å kategorisere hva som er gjort til nå innenfor området og se om noe mangler
%Kort introduksjon til Agile Software Development, Coordination og Large-scale før gjennomgang av funn i forhold til problemstillingen (som kombinerer disse emnene)

Because of the time constraints put on the research project it is obvious that some attention must be aimed towards the scope of the report and the limitations this implies. As mentioned in the previous subsection large-scale agile projects, and agile projects in general, are growing in numbers. With this growth a lot of questions and interesting research problems arise. This research project only aims to cover the described research question.

Further, the research project does not aspire to introduce a brand new theory regarding the combination of large-scale, agile software development, coordination and efficiency. The objective is to find and categorise research performed concerning the combination of these themes and look for common conclusions in their findings, as well as identifying and calling attention to clear gaps that need to be filled in the research field.

To give some insight and a clearer picture of the study theory from agile software development, coordination and large-scale will be presented. Findings from a literature search will also be given on the combination of the aforementioned themes. It is important to note that the focus on coordination will primarily be on coordination across teams and not on coordination within these teams.

\section{Report Outline}